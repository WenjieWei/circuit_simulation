\documentclass[a4paper,titlepage, 12pt]{article}
\usepackage[utf8]{inputenc}
\usepackage{fullpage}
\usepackage{indentfirst}
\usepackage[per-mode=symbol]{siunitx}
\usepackage{listings}
\usepackage{graphicx}
\usepackage{color}
\usepackage{amsmath}
\usepackage{array}
\usepackage[hidelinks]{hyperref}
\usepackage[format=plain,font=it]{caption}
\usepackage{subcaption}
\usepackage{standalone}
\usepackage[nottoc]{tocbibind}
\usepackage[noabbrev,capitalize,nameinlink]{cleveref}
\usepackage{listings}
\usepackage{xspace}
\usepackage{tikz}
\usepackage{circuitikz}
\usepackage{titlesec}
\usepackage[cache=false]{minted}
\usepackage{booktabs}
\usepackage{csvsimple}
\newcommand{\MATLAB}{\textsc{Matlab}\xspace}
\usepackage{siunitx}
\usepackage[super]{nth}
\usepackage[titletoc]{appendix}

% Custom commands
\newcommand\numberthis{\addtocounter{equation}{1}\tag{\theequation}}
\newcommand{\code}[1]{\texttt{#1}}
\newcolumntype{P}[1]{>{\centering\arraybackslash}p{#1}}

\setminted{linenos,breaklines,fontsize=auto}

%\titleformat*{\section}{\normalsize\bfseries}
%\titleformat*{\subsection}{\small\bfseries}
\renewcommand{\thesubsection}{\thesection.\alph{subsection}}
\providecommand*{\listingautorefname}{Listing}
\newcommand*{\Appendixautorefname}{Appendix}


\begin{document}
	\sloppy	
	
	\begin{center}
		{\LARGE \bf ECSE 597: Circuit Simulations and Modeling}\\
		{\large Assignment 5a, \quad \today}\\
		{\large Wenjie Wei, 260685967}
	\end{center}

	\section{Question 1}
		\subsection{Step Size of Forward Euler}
			To use forward Euler, if the pole $\lambda$ is real, we use the formula 
			\begin{equation}
				|1 + h\lambda| < 1 \quad \rightarrow \quad -2 < h\lambda < 0
				\label{feuler_equ}
			\end{equation}
			to find the appropriate size of each step, leading to the following equations: 
			
			\begin{itemize}
				\item $\lambda = -5$, (\ref{feuler_equ}) $\rightarrow$ $0 < h < \frac{2}{5}$
								
				\item $\lambda = -3$, (\ref{feuler_equ}) $\rightarrow$ $0 < h < \frac{2}{3}$
				
				\item $\lambda = -2$, (\ref{feuler_equ}) $\rightarrow$ $0 < h < 1$
				
				\item $\lambda = -7$, (\ref{feuler_equ}) $\rightarrow$ $0 < h < \frac{2}{7}$
				
				\item $\lambda = -8$, (\ref{feuler_equ}) $\rightarrow$ $0 < h < \frac{1}{4}$
			\end{itemize}
		
			Choose the smallest value here, and $(0, \frac{1}{4})$ is an appropriate interval of $h$. 
			
		\subsection{Step Size of Backward Euler}
			Due to the stability of Backward Euler method, the system is guaranteed to be stable as long as Re\{$\lambda$\} $< 0$. Since all poles given in the question have negative real values, the Backward Euler system is stable regardless of the step size. 
			
		\subsection{Trapezoidal Rule}
			The system solved by Trapezoidal Rule is stable when 
			$$
				Re\{h\lambda\} < 0
			$$
			
			Since all poles given by the question are negative, this system with step size of $h > 0$ is guaranteed to be stable. 

	\section{Question 2}
		Use the Adam-Moulton integration method shown in Equation \ref{AM_Integration} to derive a difference equation of MNA. 
		\begin{equation}
			x_n = x_{n-1} + \frac{5h}{12}\dot{x}_n + \frac{8h}{12}\dot{x}_{n-1} + \frac{h}{12}\dot{x}_{n-2}
			\label{AM_Integration}
		\end{equation}
		
		From Equation \ref{AM_Integration}, change the form:
		\begin{equation}
			5\dot{x}_{n} + 8\dot{x}_{n-1} + \dot{x}_{n-2} = \frac{12}{h}(x_n - x_{n-1})
			\label{AM_Transformed}
		\end{equation}
		
		From the MNA equation 
		$$
			\boldsymbol{G}\boldsymbol{x}_n + \boldsymbol{C}\boldsymbol{\dot{x}}_n + \boldsymbol{f}(\boldsymbol{x}_n)= \boldsymbol{b}_n
		$$
		for every $\boldsymbol{x}_n$. Replace $n$ by $n-1$ and $n-2$, and integrate the terms of $\boldsymbol{G}$, $\boldsymbol{C}$, and $\boldsymbol{b}$:
		\begin{equation}
			\boldsymbol{G}(\boldsymbol{x_n} + \boldsymbol{x_{n-1}} + \boldsymbol{x_{n-2}}) + \boldsymbol{C}(\boldsymbol{\dot{x}_n} + \boldsymbol{\dot{x}_{n-1}} + \boldsymbol{\dot{x}_{n-2}}) + \boldsymbol{F}= \boldsymbol{b}_n + \boldsymbol{b}_{n - 1} + \boldsymbol{b}_{n-2}
			\label{MNA_n}
		\end{equation}
		where 
		$$
			\boldsymbol{F} = \boldsymbol{f}(\boldsymbol{x}_n) + \boldsymbol{f}(\boldsymbol{x}_{n - 1}) + \boldsymbol{f}(\boldsymbol{x}_{n - 2})
		$$
		
		Plug Equation \ref{AM_Transformed} into \ref{MNA_n}, and we can arrive at:
		\begin{align*}
			\boldsymbol{G}(\boldsymbol{x_n} + \boldsymbol{x_{n-1}} + \boldsymbol{x_{n-2}}) + \frac{12C}{h}(\boldsymbol{x}_n + \boldsymbol{x}_{n - 1}) + \boldsymbol{F} &= 5\boldsymbol{b}_n + 8\boldsymbol{b}_{n-1} + \boldsymbol{b}_{n-2}\\
			(5\boldsymbol{G} + \frac{12\boldsymbol{C}}{h})\boldsymbol{x}_n + (8\boldsymbol{G} - \frac{12\boldsymbol{C}}{h})\boldsymbol{x}_{n-1} + \boldsymbol{G}\boldsymbol{x}_{n - 2} + \boldsymbol{F} &= 5\boldsymbol{b}_n + 8\boldsymbol{b}_{n-1} + \boldsymbol{b}_{n-2}
		\end{align*}
		which is the solution difference equation for this question. 
		
	\section{Question 3}
		For simplicity, use LU to denote the LU decomposition and F/B to denote the forward / backward substitution. Table below shows the computational cost of each method, with steps followed. 
		
		\begin{table}[H]
			\begin{center}
				\begin{tabular}{c|c|c|c}
						& Sparse Ordering & LU & F/B\\\hline
					Perturbation & 1 & 4000 & 4000\\\hline
					Differentiation & 1 & 1000 & 4000\\\hline
					Adjoint & 1 & 1000 & 6000
				\end{tabular}
				\label{Solution}
			\end{center}
		\end{table}
		\subsection{Perturbation Method}
			\begin{itemize}
				\item 1 sparse ordering for the entire process;
				\item 1 LU and 1 F/B to find the frequency response;
				\item According to the equation
				$$
					(A + \Delta A)(x + \Delta x) = b
				$$
				A is changing for each sensitivity test, therefore, 1 additional LU and F/B is required to find the sensitivity with respect to every additional parameter.
			\end{itemize}
			
			Therefore, the Perturbation method will need 1 sparse ordering. 1000 frequency points result in 1000 LU and F/B, and the three sensitivity parameters lead to an additional 3000 LU and F/B calculations. 
		
		\subsection{Differentiation Method}
			\begin{itemize}
				\item 1 sparse ordering for the entire process;
				\item Matrix $\boldsymbol{A}$ is reused at each frequency point, thus only 1 LU is needed. But the sensitivity with respect to the frequency requires an LU and F/B at each frequency points. 
				\item An additional F/B is needed to compute sensitivity with respect to each parameter. 
			\end{itemize}
		
			Therefore, the Differentiation method requires 1 sparse ordering, 1000 LU and F/B for overall sensitivity and frequency response, and an additional 3000 F/B for each sensitivity parameter. 
			
		\subsection{Adjoint Method}
			\begin{itemize}
				\item 1 sparse ordering for the entire process;
				\item 1 LU and 1 F/B for sensitivity at each frequency point;
				\item 1 additional F/B for sensitivity analysis with respect to all output nodes. 
			\end{itemize}
			
			Therefore, the Adjoint method requires 1 sparse ordering, 1000 LU and F/B for overall sensitivity and frequency response, and an additional 5000 F/B for 5 output nodes. 
\end{document}























