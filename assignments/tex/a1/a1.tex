\documentclass[a4paper,titlepage]{article}
\usepackage[utf8]{inputenc}
\usepackage{fullpage}
\usepackage{indentfirst}
\usepackage[per-mode=symbol]{siunitx}
\usepackage{listings}
\usepackage{graphicx}
\usepackage{color}
\usepackage{amsmath}
\usepackage{array}
\usepackage[hidelinks]{hyperref}
\usepackage[format=plain,font=it]{caption}
\usepackage{subcaption}
\usepackage{standalone}
\usepackage[nottoc]{tocbibind}
\usepackage[noabbrev,capitalize,nameinlink]{cleveref}
\usepackage{listings}
\usepackage{xspace}
\usepackage{tikz}
\usepackage{circuitikz}
\usepackage{titlesec}
\usepackage[cache=false]{minted}
\usepackage{booktabs}
\usepackage{csvsimple}
\newcommand{\MATLAB}{\textsc{Matlab}\xspace}
\usepackage{siunitx}
\usepackage[super]{nth}
\usepackage[titletoc]{appendix}

% Custom commands
\newcommand\numberthis{\addtocounter{equation}{1}\tag{\theequation}}
\newcommand{\code}[1]{\texttt{#1}}
\newcolumntype{P}[1]{>{\centering\arraybackslash}p{#1}}

\setminted{linenos,breaklines,fontsize=auto}

%\titleformat*{\section}{\normalsize\bfseries}
%\titleformat*{\subsection}{\small\bfseries}
\renewcommand{\thesubsection}{\thesection.\alph{subsection}}
\providecommand*{\listingautorefname}{Listing}
\newcommand*{\Appendixautorefname}{Appendix}

%opening
\title{\textbf{ECSE 597: Circuit Simulation and Modeling} \\ Assignment 1}
\author{Wenjie Wei \\ 260685967}
\date{\today}

\begin{document}
	\sloppy	
	
	\begin{center}
		{\LARGE \bf ECSE 597: Circuit Simulations and Modeling}\\
		{\large Assignment 1, Sept. 10, 2019}\\
		{Wenjie Wei, 260685967}
	\end{center}
	
	\section{Question 1}
		Matrix $A$ is given as following:
		$$
			A = \begin{bmatrix}
				2 & 4 & 6 & 2 & 1\\
				4 & 9 & 14 & 8 & 8\\
				2 & 6 & 12 & 12 & 21\\
				4 & 10 & 20 & 17 & 32\\
				8 & 17 & 30 & 18 & 34
			\end{bmatrix}.
		$$
	
		\subsection{Doolittle's Algorithm}		
			To perform the LU decomposition using Doolittle's algorithm, first row of $U$ and first column of $L$ can be calculated by:
			\begin{equation}
				u_{1k} = a_{1k}, \quad k = 1, ..., n
				\label{1st_row}
			\end{equation}		
			and
			\begin{equation}
				l_{k1} = a_{k1} / u_{11}, \quad k = 2,...,n
				\label{1st_col}
			\end{equation}
			
			The following equations will be applied to compute the LU decomposition using Doolittle's algorithm:
			\begin{equation}
				u_{ik} = a_{ik} - \sum_{m=1}^{i-1} l_{im}u_{mk}, \quad k = i, ..., n
				\label{rows}
			\end{equation}
			\begin{equation}
				l_{kj} = (a_{kj} - \sum_{m=1}^{j-1} l_{km}u_{mj}) / u_{jj}, \quad k = (j+1), ..., n
				\label{cols}
			\end{equation}
			
			From Equation \ref{1st_row} and \ref{1st_col}, and define that the diagonal entries of $L$ are 1, we calculate the \textbf{\textit{first row}} of $U$ and the \textbf{\textit{first column}} of $L$:
			$$
			U = \begin{bmatrix}
				2 & 4 & 6 & 2 & 1\\
				0 & u_{22} & u_{23} & u_{24} & u_{25}\\
				0 & 0 & u_{33} & u_{34} & u_{35}\\
				0 & 0 & 0 & u_{44} & u_{45}\\
				0 & 0 & 0 & 0 & u_{55}
			\end{bmatrix}, \quad
			L = \begin{bmatrix}
				1 & 0 & 0 & 0 & 0\\
				2 & 1 & 0 & 0 & 0\\
				1 & l_{32} & 1 & 0 & 0\\
				2 & l_{42} & l_{43} & 1 & 0\\
				4 & l_{52} & l_{53} & l_{54} & 1
			\end{bmatrix}
			$$
			
			Use Equation \ref{rows} and \ref{cols}, and from the $U$ and $L$ obtained above, calculate the rest entries of $L$ and $U$:
			
			Second Step: Use Equation \ref{rows}, $u_{22}$ can be calculated as such:
			\begin{align*}
				u_{22} &= a_{22} - \sum_{m = 1}^{1}l_{2m}u_{12}\\
				&= 9 - 2\times4 = 1
			\end{align*}
			\begin{align*}
				l_{32} &= (a_{32} - \sum_{m=1}^{1}l_{3m}u_{m2})/u_{22}\\
				&= (6 - 1\times 4) / 1 = 2
			\end{align*}
			Use the similar techniques, the \textbf{\textit{second row and column}} of $U$ and $L$ are now updated as such:
			$$
				U = \begin{bmatrix}
					2 & 4 & 6 & 2 & 1\\
					0 & 1 & 2 & 4 & 6\\
					0 & 0 & u_{33} & u_{34} & u_{35}\\
					0 & 0 & 0 & u_{44} & u_{45}\\
					0 & 0 & 0 & 0 & u_{55}
				\end{bmatrix}, \quad
				L = \begin{bmatrix}
					1 & 0 & 0 & 0 & 0\\
					2 & 1 & 0 & 0 & 0\\
					1 & 2 & 1 & 0 & 0\\
					2 & 2 & l_{43} & 1 & 0\\
					4 & 1 & l_{53} & l_{54} & 1
				\end{bmatrix}
			$$
			
			Continue using the same equations, and obtain the \textbf{\textit{third row and column}} of $U$ and $L$:
			
			$$
				U = \begin{bmatrix}
					2 & 4 & 6 & 2 & 1\\
					0 & 1 & 2 & 4 & 6\\
					0 & 0 & 2 & 2 & 8\\
					0 & 0 & 0 & u_{44} & u_{45}\\
					0 & 0 & 0 & 0 & u_{55}
				\end{bmatrix}, \quad
				L = \begin{bmatrix}
					1 & 0 & 0 & 0 & 0\\
					2 & 1 & 0 & 0 & 0\\
					1 & 2 & 1 & 0 & 0\\
					2 & 2 & 2 & 1 & 0\\
					4 & 1 & 2 & l_{54} & 1
				\end{bmatrix}
			$$
			
			Continue using the same equations, and obtain the \textbf{\textit{forth row and column}} of $U$ and $L$:
			
			$$
			U = \begin{bmatrix}
			2 & 4 & 6 & 2 & 1\\
			0 & 1 & 2 & 4 & 6\\
			0 & 0 & 2 & 2 & 8\\
			0 & 0 & 0 & 1 & 2\\
			0 & 0 & 0 & 0 & u_{55}
			\end{bmatrix}, \quad
			L = \begin{bmatrix}
			1 & 0 & 0 & 0 & 0\\
			2 & 1 & 0 & 0 & 0\\
			1 & 2 & 1 & 0 & 0\\
			2 & 2 & 2 & 1 & 0\\
			4 & 1 & 2 & 2 & 1
			\end{bmatrix}
			$$
			
			And finally, compute $u_{55}$ to obtain the \textbf{\textit{complete}} decomposition of $A$:
			
			$$
			U = \begin{bmatrix}
			2 & 4 & 6 & 2 & 1\\
			0 & 1 & 2 & 4 & 6\\
			0 & 0 & 2 & 2 & 8\\
			0 & 0 & 0 & 1 & 2\\
			0 & 0 & 0 & 0 & 4
			\end{bmatrix}, \quad
			L = \begin{bmatrix}
			1 & 0 & 0 & 0 & 0\\
			2 & 1 & 0 & 0 & 0\\
			1 & 2 & 1 & 0 & 0\\
			2 & 2 & 2 & 1 & 0\\
			4 & 1 & 2 & 2 & 1
			\end{bmatrix}
			$$
			
			Verify this result by doing a cross product, and it is verified that 
			$$
				L \times U = A.
			$$
			
		\subsection{Gaussian Version of Doolittle's Algorithm}
			Use Gaussion steps to perform the Doolittle's algorithm. 
			
			Choose $a_{11}$ as the pivot as the first step, and divide every entry of $A$ below the pivot by the pivot, we obtain that
			
			$$
				A = \begin{bmatrix}
					\textcolor{red}{2} & 4 & 6 & 2 & 1\\
					4 & 9 & 14 & 8 & 8\\
					2 & 6 & 12 & 12 & 21\\
					4 & 10 & 20 & 17 & 32\\
					8 & 17 & 30 & 18 & 34
				\end{bmatrix} \rightarrow \begin{bmatrix}
					\textcolor{red}{2} & 4 & 6 & 2 & 1\\
					\textbf{2} & 9 & 14 & 8 & 8\\
					\textbf{1} & 6 & 12 & 12 & 21\\
					\textbf{2} & 10 & 20 & 17 & 32\\
					\textbf{4} & 17 & 30 & 18 & 34
				\end{bmatrix}
			$$
			
			After this step, overwrite each entry from $a_{22}$ to $a_{55}$, such that 
			$$
				a_{ij} \rightarrow a_{ij} - a_{ik} \times a_{kj}
			$$
			where $k$ stands for the row index of the pivot. 
			
			After this step, we obtain a new $A$ matrix as shown below:
			$$
				A = \begin{bmatrix}
					2 & 4 & 6 & 2 & 1\\
					2 & 9 & 14 & 8 & 8\\
					1 & 6 & 12 & 12 & 21\\
					2 & 10 & 20 & 17 & 32\\
					4 & 17 & 30 & 18 & 34
				\end{bmatrix} \rightarrow \begin{bmatrix}				
					2 & 4 & 6 & 2 & 1\\
					2 & 1 & 2 & 4 & 6\\
					1 & 2 & 6 & 10 & 20\\
					2 & 2 & 8 & 13 & 30\\
					4 & 1 & 6 & 10 & 30
				\end{bmatrix}
			$$
			
			Move the pivot to $a_{22}$, and perform the same procedures like what is done above. 
			
			Step 2:
			$$
				A = \begin{bmatrix}				
					2 & 4 & 6 & 2 & 1\\
					2 & \textcolor{red}{1} & 2 & 4 & 6\\
					1 & 2 & 6 & 10 & 20\\
					2 & 2 & 8 & 13 & 30\\
					4 & 1 & 6 & 10 & 30
				\end{bmatrix} \rightarrow \begin{bmatrix}				
					2 & 4 & 6 & 2 & 1\\
					2 & \textcolor{red}{1} & 2 & 4 & 6\\
					1 & 2 & 6 & 10 & 20\\
					2 & 2 & 8 & 13 & 30\\
					4 & 1 & 6 & 10 & 30
				\end{bmatrix} \rightarrow \begin{bmatrix}				
					2 & 4 & 6 & 2 & 1\\
					2 & \textcolor{red}{1} & 2 & 4 & 6\\
					1 & 2 & 2 & 2 & 8\\
					2 & 2 & 4 & 5 & 18\\
					4 & 1 & 4 & 6 & 24
				\end{bmatrix}
			$$
			
			Step 3:
			$$
			A = \begin{bmatrix}				
					2 & 4 & 6 & 2 & 1\\
					2 & 1 & 2 & 4 & 6\\
					1 & 2 & \textcolor{red}{2} & 2 & 8\\
					2 & 2 & 4 & 5 & 18\\
					4 & 1 & 4 & 6 & 24
				\end{bmatrix} \rightarrow \begin{bmatrix}				
					2 & 4 & 6 & 2 & 1\\
					2 & 1 & 2 & 4 & 6\\
					1 & 2 & \textcolor{red}{2} & 2 & 8\\
					2 & 2 & 2 & 5 & 18\\
					4 & 1 & 2 & 6 & 24
				\end{bmatrix} \rightarrow \begin{bmatrix}				
					2 & 4 & 6 & 2 & 1\\
					2 & 1 & 2 & 4 & 6\\
					1 & 2 & \textcolor{red}{2} & 2 & 8\\
					2 & 2 & 2 & 1 & 2\\
					4 & 1 & 2 & 2 & 8
				\end{bmatrix}
			$$
			
			Step 4:
			$$
				A = \begin{bmatrix}				
					2 & 4 & 6 & 2 & 1\\
					2 & 1 & 2 & 4 & 6\\
					1 & 2 & 2 & 2 & 8\\
					2 & 2 & 2 & \textcolor{red}{1} & 2\\
					4 & 1 & 2 & 2 & 8
				\end{bmatrix} \rightarrow \begin{bmatrix}				
					2 & 4 & 6 & 2 & 1\\
					2 & 1 & 2 & 4 & 6\\
					1 & 2 & 2 & 2 & 8\\
					2 & 2 & 2 & \textcolor{red}{1} & 2\\
					4 & 1 & 2 & 2 & 8
				\end{bmatrix} \rightarrow \begin{bmatrix}				
					2 & 4 & 6 & 2 & 1\\
					2 & 1 & 2 & 4 & 6\\
					1 & 2 & 2 & 2 & 8\\
					2 & 2 & 2 & \textcolor{red}{1} & 2\\
					4 & 1 & 2 & 2 & 4
				\end{bmatrix}
			$$
			
			After these processes, we obtain the matrices $L$ and $U$ in separated format:
			$$
			L = \begin{bmatrix}
			1 & 0 & 0 & 0 & 0\\
			2 & 1 & 0 & 0 & 0\\
			1 & 2 & 1 & 0 & 0\\
			2 & 2 & 2 & 1 & 0\\
			4 & 1 & 2 & 2 & 1
			\end{bmatrix}, \quad		
			U = \begin{bmatrix}
			2 & 4 & 6 & 2 & 1\\
			0 & 1 & 2 & 4 & 6\\
			0 & 0 & 2 & 2 & 8\\
			0 & 0 & 0 & 1 & 2\\
			0 & 0 & 0 & 0 & 4
			\end{bmatrix}		
			$$
			
			This result agrees with the results obtained by Doolittle's algorithm, and it is verified that 
			$$
				L\times U = A.
			$$
		\subsection{Similarities and Differences of the Methods}
			Similarities:
			\begin{itemize}
				\item Both based on Gaussian Elimination of matrices; 
				\item Both methods start their iterations from the diagonal entry, \textit{i.e.} the pivot, and all calculations for the entries are based on the value of the pivot.
			\end{itemize}
		
			Differences:
			\begin{itemize}
				\item Gaussian Version of Doolittle's algorithm is simpler to implement than the Doolittle's algorithm, as there are fewer summation for each iteration. 
				\item Gaussian Version of Doolittle's algorithm calculates all entries after each iteration. This can later result in a longer calculation time. Doolittle's algorithm calculates the entries in the respective rows and columns only. 
			\end{itemize}
\end{document}


















